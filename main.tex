\documentclass{article}

\usepackage[T1]{fontenc}
\usepackage[utf8]{inputenc}
\usepackage{lmodern}
\usepackage{kotex}

\usepackage[a4paper, landscape, total={10in, 7in}]{geometry}
\usepackage{lipsum}

\usepackage{minted}

\usepackage{multicol}

\begin{document}
	
\begin{multicols}{2}
	
\title{2022 ACM-ICPC Teamnote}
\author{HeukseokZZANG}
\maketitle

\tableofcontents

\section{기본 템플릿}
\inputminted{cpp}{codes/basic.cpp}

\section{주요 알고리즘}
\subsection{유니온 파인드}
\inputminted{cpp}{codes/unionfind.cpp}
\subsection{다익스트라}
\inputminted{cpp}{codes/dijkstra.cpp}
\subsection{DFS}
\inputminted{cpp}{codes/dfs.cpp}
\subsection{BFS}
\inputminted{cpp}{codes/bfs.cpp}
\subsection{선분 교차 판정}
\inputminted{cpp}{codes/linecross.cpp}
\subsection{소수 리스트 생성}
\inputminted{python}{codes/primelist.py}
\subsection{소수 판정 알고리즘}
\inputminted{python}{codes/primetest.py}
\subsection{밀러-라빈 소수 판정}
\inputminted{python}{codes/miller-rabin.py}
\subsection{폴라드-로 소인수분해}
\inputminted{python}{codes/pollard-rho.py}

\section{수학}
\subsection{NTT}
\inputminted{python}{codes/ntt.py}
\subsection{스프라그-그런디}
\inputminted{python}{codes/nim.py}
\subsection{유클리드 호제법}
\inputminted{cpp}{codes/euclid.cpp}
\subsection{확장 유클리드}
\inputminted{python}{codes/eea.py}
\subsection{페르마 소정리}

\subsection{중국인의 나머지 정리}
\subsection{모듈러 곱셈 역원}
\inputminted{python}{codes/modular.py}
\subsection{좌표 압축}
\inputminted{python}{codes/comp.py}

\section{그래프}
\subsection{최대 유량}
\inputminted{python}{codes/mf.py}
\subsection{이분 매칭}
\inputminted{python}{codes/bimatch.py}

\section{트리}
\subsection{세그먼트 트리}
\inputminted{cpp}{codes/segtree.cpp}
\subsection{펜윅 트리}
\inputminted{python}{codes/fenwick.py}
\subsection{2차원 펜윅 트리}
\inputminted{python}{codes/fenwick2d.py}

\section{테크닉}
\subsection{비트마스킹}
\subsection{이분탐색}
\inputminted{python}{codes/bisect.py}


\end{multicols}
%\lipsum
%
%\section{Hello World!}
%\begin{minted}{c}
%	int main() {
%		printf("hello, world");
%		return 0;
%	}
%\end{minted}
%
%\section{Math in Source Code Comments}
%\begin{minted}[mathescape,gobble=2]{csharp}
%	/*
%	Defined as $\pi=\lim_{n\to\infty}\frac{P_n}{d}$ where $P$ is the perimeter
%	of an $n$-sided regular polygon circumscribing a
%	circle of diameter $d$.
%	*/
%	const double pi = 3.1415926535
%\end{minted}
	
\end{document}